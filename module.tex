\documentclass[11pt,a4paper]{article}
\usepackage[xetex]{hyperref}
\usepackage{mathtools,latexsym,amsfonts,amssymb,MnSymbol}
\usepackage{xcolor}
\usepackage{xifthen}
\usepackage{fancyhdr}
\usepackage{ccicons}

\hypersetup{colorlinks=true}
\pagestyle{fancy}

\fancyhf{}
\cfoot{\thepage}
\rfoot{\ccby}
\renewcommand{\headrulewidth}{0.0pt}
\renewcommand{\footrulewidth}{0.4pt}

\newcommand{\key}[1]{\textrm{\textbf{#1}}}
\newcommand{\sig}       {\key{sig}}
\newcommand{\End}       {\key{end}}
\newcommand{\val}       {\key{val}}
\newcommand{\type}      {\key{type}}
\newcommand{\eqtype}    {\key{eqtype}}
\newcommand{\datatype}  {\key{datatype}}
\newcommand{\exception} {\key{exception}}
\newcommand{\of}        {\key{of}}
\newcommand{\inc}       {\key{include}}
\newcommand{\sharing}   {\key{sharing}}
\newcommand{\where}     {\key{where}}
\newcommand{\typerfn}   {\textit{typerfn}}
\newcommand{\Let}       {\key{let}}
\newcommand{\Structure} {\key{structure}}
\newcommand{\Struct}	{\key{struct}}
\newcommand{\In}        {\key{in}}
\newcommand{\Local}     {\key{local}}
\newcommand{\signature} {\key{signature}}
\newcommand{\functor}   {\key{functor}}

\newcommand{\B}  {\textrm{B}}
\newcommand{\sB} {\textrm{\tiny B}}
\newcommand{\E}  {\textrm{E}}
\newcommand{\sE} {\textrm{\tiny E}}
\newcommand{\T}  {\textrm{T}}
\newcommand{\sT} {\textrm{\tiny T}}
\newcommand{\TE} {\textrm{TE}}
\newcommand{\sTE}{\textrm{\tiny TE}}
\newcommand{\VE} {\textrm{VE}}
\newcommand{\sVE}{\textrm{\tiny VE}}
\newcommand{\SE} {\textrm{SE}}
\newcommand{\sSE}{\textrm{\tiny SE}}
\renewcommand{\G}{\textrm{G}}
\newcommand{\sG} {\textrm{\tiny G}}
\newcommand{\F}  {\textrm{F}}
\newcommand{\sF} {\textrm{\tiny F}}

\newcommand{\tycon}{\mathbb{C}}
\newcommand{\vcon} {\mathbf{C}}
\newcommand{\equality}{\mathcal{E}}
\newcommand{\eqyes}{\equiv}
\newcommand{\eqnot}{\nequiv}
\newcommand{\VKE}  {\textrm{EXN}}
\newcommand{\VKV}  {\textrm{VAR}}
\newcommand{\VKC}  {\textrm{CONS}}
\newcommand{\corenew}[1]{\textit{new\_#1}}

\newcommand{\vdashSigBind} {\ \vdash_{\textrm{\tiny sigbind}}\ }
\newcommand{\vdashSigexp}  {\ \vdash_{\textrm{\tiny sigexp}}\ }
\newcommand{\vdashSpec}    {\ \vdash_{\textrm{\tiny spec}}\ }
\newcommand{\vdashSpecTB}  {\ \vdash_{\textrm{\tiny specTB}}\ }
\newcommand{\vdashSpecDBA} {\ \vdash_{\textrm{\tiny specDB1}}\ }
\newcommand{\vdashSpecDBB} {\ \vdash_{\textrm{\tiny specDB2}}\ }
\newcommand{\vdashTyperfn} {\ \vdash_{\textrm{\tiny rfn}}\ }
\newcommand{\vdashShareTycon}{\ \vdash_{\textrm{\tiny share-tycon}}\ }
\newcommand{\vdashShareStr}	{\ \vdash_{\textrm{\tiny share-str}}\ }
\newcommand{\vdashStr}		{\ \vdash_{\textrm{\tiny str}}\ }
\newcommand{\vdashStrDec}	{\ \vdash_{\textrm{\tiny strdec}}\ }
\newcommand{\vdashStrBind} {\ \vdash_{\textrm{\tiny strbind}}\ }
\newcommand{\vdashFctBind} {\ \vdash_{\textrm{\tiny fctbind}}\ }
\newcommand{\vdashFctArg}  {\ \vdash_{\textrm{\tiny fctarg}}\ }
\newcommand{\vdashFctParm} {\ \vdash_{\textrm{\tiny fctparm}}\ }
\newcommand{\vdashProg}	   {\ \vdash_{\textrm{\tiny prog}}\ }
\newcommand{\vdashT}       {\ \vdash_{\textrm{\tiny T}}\ }
\newcommand{\vdashD}       {\ \vdash_{\textrm{\tiny D}}\ }
\newcommand{\braced}[1]{\{#1\}}
\newcommand{\angled}[1]{\,{\color{gray}\langle#1\rangle}\,}
\newcommand{\qualtype}[2]{#1 \triangleright #2}
\newcommand{\tyfun}[2]{\mathbf{\Lambda} #1\,.\,#2}
\newcommand{\fctor}[2]{#1 \leadsto #2}

\newcommand{\MaximizeEq}{\key{ME}}
\newcommand{\matchEnv}[3][]{ \ifthenelse{\isempty{#1}}{#2 \sqsubseteq #3}{#1 \Leftarrow #2 \sqsubseteq #3}}
\newcommand{\SUPP}[1]{\textrm{SUPP}(#1)}
\newcommand{\KEYS}[1]{keys(#1)}
\newcommand{\TYNAMES}[1]{\KEYS{#1}}
\newcommand{\GENTYPES}[2]{renewtypes_{\textrm{\tiny $#1$}}(#2)}
\newcommand{\compose}[2]{#1 \circ #2}
\newcommand{\wildcard}{{\color{gray}\scriptstyle\textbf{*}}}
\newcommand{\Empty}{\braced{}}
\newcommand{\absval}{\star}

\begin{document}

\title {A pure specification of MLton - Part 3. Module}
\author{Wu Jiasen $\langle$\href{mailto:wujiasen@yahoo.com}{wujiasen@yahoo.com}$\rangle$}
\maketitle 
\thispagestyle{fancy}

\section{Modeling the Module}
\[\begin{array}{lcl}
    \B & = & (\F, \G, \E) \\
    \F & = & \braced{fctid \mapsto (\fctor{\E_1^{\sT_1}}{\E_2^{\sT_2}}, coredecs)} \\
    \G & = & \braced{sigid \mapsto \E^\sT} \\
    \E & = & (\TE,\SE,\VE)  \\
    \TE& = & \braced{tycon \mapsto (arity, tyfun, constructors)} \\
    \SE& = & \braced{strid \mapsto \E}  \\ 
    \VE& = & \braced{vid   \mapsto (coreval,\delta, kind} \\
    kind & = & \VKV | \VKC | \VKE \\
\end{array}\]

\section{Signature}
{\renewcommand{\arraystretch}{1.2}\[
\begin{array}{lcl}
sigbind
    & = & \overline{sigid = sigexp}                        \\

sigexp 
    & = & sigid \\
    & | & \sig\ spec\ \End  \\
    & | & sigexp\ \where\ \overline{\type\ \overline{tyvar}\ tycon = type} \\
    
spec 
    & = & \val\ \overline{vid : type} \\
    & | & \type\ \overline{\overline{tyvar}\ tycon} \\
    & | & \eqtype\ \overline{\overline{tyvar}\ tycon} \\
    & | & \type\ \overline{\overline{tyvar}\ tycon = type} \\
    & | & \datatype\ tycon = \datatype longtycon \\
    & | & \datatype\ \overline{\overline{tyvar}\ tycon = \overline{vid\ \of\ type}} \\
    & | & \exception\ \overline{vid\ \of\ type} \\
    & | & \Structure\ \overline{strid:sigexp} \\
    & | & \inc\ \overline{sigid} \\
    & | & \inc\ sigexp \\
    & | & spec\ \sharing\ \type\ \overline{longtycon} \\
    & | & spec\ \sharing\ \overline{longstrid} \\
    & | & spec\ \key{;}\ spec \\
\end{array}\]}

\section{Elaboration of signature}
\vspace{-25pt}
\begin{flushright}
\framebox{ $ \B, sigbind \vdashSigBind \B'$ }
\end{flushright}


\[
\frac
 {\B, sigexp \vdashSigexp \E \qquad
  \T' = \TYNAMES{\E}-\TYNAMES{\B}}
 {\B, sigid = sigexp \vdashSigBind \braced{sigid \mapsto \E^{\sT'}}^\sG}
\]

\[
\frac
 { sigid \mapsto \E^{\sT} \in B }
 {\B, sigid \vdashSigexp \E}
\]

\[
\frac
 {\B, spec \vdashSpec \E}
 {\B, \sig\ spec\ \End \vdashSigexp \E}
\]

\[
\frac
 {\begin{array}{c}
  \B, sigexp \vdashSigexp \E \qquad
  \B, \E, \typerfn_i \vdashTyperfn \varphi_i
  \end{array}}
 {\B, sigexp\ \where\ \overline{\typerfn} \vdashSigexp [\sum_i \varphi_i]\E}
\]

\subsection{elaborating specification}
\[
\cfrac
 {\B, type_i \vdashT \qualtype{p_i}{\tau_i}}
 {\B, \val\ \overline{vid:type}_{1..n} \vdashSpec \sum_i \braced{vid_i \mapsto (\absval, gen_\sB(\qualtype{p_i}{\tau_i}), \VKV)}}
\]

\[
\cfrac
 {\overline{\alpha}_i^{(k_i)} = tyvars_i \qquad
  \tycon_i \Leftarrow \corenew{tycon}}
 {\B, \type\ \overline{tyvars\ tycon} \vdashSpec \sum_i \braced{tycon_i \mapsto (k_i, \tycon_i^\eqnot)}^\sTE}
\]

\[
\cfrac
 {\overline{\alpha}_i^{(k_i)} = tyvars_i \qquad
  \tycon_i \Leftarrow \corenew{tycon}}
 {\B, \eqtype\ \overline{tyvars\ tycon} \vdashSpec \sum_i \braced{tycon_i \mapsto (k_i, \tycon_i^\eqyes)}^\sTE}
\]

\[
\cfrac
 {\B, tybind_i \vdashSpecTB \TE_i}
 {\B, \type\ \overline{tybind} \vdashSpec \sum_i \TE_i}
\]

\[
\cfrac
 {\begin{array}{c}
  (eqset,\wildcard) \Leftarrow \MaximizeEq(\overline{databind}) \qquad
  \B, eqset, databind_i \vdashSpecDBA \TE_i             \\
  \B + \sum_i \TE_i, databind_j \vdashSpecDBB \VE_j
  \end{array}}
 {\B, \datatype\ \overline{databind} \vdashSpec \sum_i \TE_i \oplus \sum_j \VE_j}
\]

\[
\cfrac
 {\B, type_i \vdashT \qualtype{p_i}{\tau_i} \qquad
  free(\qualtype{p_i}{\tau_i}) = \emptyset
  \vcon\Leftarrow\corenew{vcon}}
 {\B, \exception\ \overline{con\ \of\ type} \vdashSpec
    \sum_i \braced{con_i \mapsto (\vcon_i, \qualtype{p_i}{\tau_i\to exn}, \VKE)}^\sVE}
\]

\[
\cfrac
 {\B, sigexp_i \vdashSigexp \E_i}
 {\B, \Structure\ \overline{strid:sigexp} \vdashSpec
   \sum_i \braced{strid_i \mapsto \E_i}^\sSE }
\]

\[
\cfrac
 {\B, sigexp \vdashSigexp \E}
 {\B, \inc\ sigexp \vdashSpec \E}
\]

\[
\cfrac
 {sigid_i \mapsto \E_i^{\T_i} \in^\sG \B}
 {\B, \inc\ \overline{sigid} \vdashSpec \sum_i \E_i}
\]

\[
\cfrac
 {\B,      spec_1 \vdashSpec \E_1 \qquad
  \B+\E_1, spec_2 \vdashSpec \E_2}
 {\B, spec_1\ \key{;}\ spec_2 \vdashSpec \E_1 + \E_2}
\]

\[
\cfrac
 { \B, spec \vdashSpec \E \qquad
   \B, \E, \overline{longtycon} \vdashShareTycon \E' }
 {\B, spec\ \sharing\ \type\ \overline{longtycon} \vdashSpec \E'}
\]

\[
\cfrac
 {\B, spec \vdashSpec \E \qquad
  \B, \E, \overline{longstrid},\braced{(i,j)\,|\,i<j}\vdashShareStr E'}
 {\B, spec\ \sharing\ \overline{longstrid} \vdashSpec \E'}
\]

\subsection {type refinement}
\[
\cfrac
 {tycon \mapsto k, \tycon^\equality \in \E                  \qquad
  \overline{\alpha}^{(k)} = \overline{tyvar}                \qquad
  \B, type \vdashT \qualtype{p}{\tycon'^\equality\ \overline{\alpha}} \qquad
  }
 {\B, \E, \type\ \overline{tyvar}\ tycon = type \vdashTyperfn \braced{\tycon \mapsto \tycon'}}
\]
\[
\cfrac
 {tycon \mapsto k, \tycon^\equality \in \E                  \qquad
  \overline{\alpha}^{(k)} = tyvars                          \qquad
  \B, type \vdashT \qualtype{p}{\tau}                       \qquad
  \equality = eq(\tau)
  }
 {\B, \E, \type\ tyvars\ tycon = type \vdashTyperfn
    \braced{\tycon \mapsto \tyfun{\overline{\alpha}}{\qualtype{p}{\tau}}}}
\]

\subsection {elaborating type bindings}
\[
\cfrac
 {\B, type \vdashT \tycon^\equality \overline\alpha }
 {\B,\overline{\alpha}^{(k)}\ tycon = type \vdashSpecTB \braced{tycon \mapsto (k, \tycon^\equality,\emptyset}^\TE}
\]
\[
\cfrac
 {\B, type \vdashT \tau \qquad free(\tau) \subseteq \overline\alpha}
 {\B,\overline{\alpha}^{(k)}\ tycon = type \vdashSpecTB 
     \braced{tycon \mapsto (k, \tyfun{\overline\alpha}{\tau},\emptyset)}^\TE}
\]
\subsection {elaborating datatype bindings}
\[
\cfrac
 {\overline{\alpha}^{(k)} = tyvars      \qquad
  \tycon \Leftarrow \corenew{tycon} 	\qquad
  \equality = \left\{\begin{array}{ll}
  			  \eqyes	& \textrm{if } tycon \in eqset  	\\
			  \eqnot	& \textrm{otherwise.}				\\
  			  \end{array}\right.
 }
 {\B,eqset,tyvars\ tycon = \overline{vid\ \angled{\key{of}\ type}} \vdashSpecDBA
    \braced{tycon \mapsto (k, \tycon^\equality,\overline{vid})}^\TE }
\]

\[
\cfrac
 {\begin{array}{c}
  \overline{\alpha}^{(k)} = tyvars                 \qquad
  tycon \mapsto (k,\tycon) \in^\TE \B 	           \\
  \angled{\B,type_i \vdashT \qualtype{p_i}{\tau_i} \qquad
  free(\tau_i) \subseteq \overline{\alpha}}        \\
  \delta_i = \forall{\overline\alpha}.(
    \qualtype{\angled{p_i}}{{\angled{\tau_i \to }\tycon\ \overline\alpha}}) \qquad
  \vcon_i\Leftarrow\corenew{vcon}			       \\
  \end{array} }
 {
  \B,tyvars\ tycon = \overline{vid\ \angled{\key{of}\ type}} \vdashSpecDBB 
    \sum_i\braced{con_i \mapsto (\vcon_i,\delta_i,\VKC)}^\VE 
 }
\]

\subsection {type sharing}
\[
\cfrac
 {\begin{array}{c}
  longtycon_i \mapsto (k_i, \tycon_i^{\equality_i}) \in^\sTE \E \qquad
  k_1 = \cdots k_n                                              \\
  \forall i. \tycon_i \not\in^\sT \B \qquad
  \equality = \left\{\begin{array}{ll}
                \eqyes & \exists i. \equality_i = \eqyes \\
                \eqnot & \textrm{otherwise.}             \\
              \end{array}\right.                                \\
  \tycon = \corenew{tycon}  \qquad
  \varphi = \sum_i \braced{\tycon_i \mapsto \tycon^\equality}
  \end{array}}
 {\B, \E, \overline{longtycon} \vdashShareTycon [\varphi]\E}
\]
\subsection {structure sharing}
\[
\cfrac
 {}
 {\B,\E,\overline{longstrid}, \emptyset \vdashShareStr \E}
\]
\[
\cfrac
 {\begin{array}{c}
  longstrid_i \mapsto \E_a \in \E   \qquad
  longstrid_j \mapsto \E_b \in \E   \qquad
  \overline{tycon}^{(k)} = \E_a \cap \E_b \\
  \begin{array}{lllll}
  \B,&\E  ,&longstrid_i.tycon_1 = longstrid_j.tycon_1 &\vdashShareTycon &\E_1 \\
  \B,&\E_1,&longstrid_i.tycon_2 = longstrid_j.tycon_2 &\vdashShareTycon &\E_2 \\
  \multicolumn{5}{c}{\cdots}\\
  \B,&\E_k,&longstrid_i.tycon_k = longstrid_j.tycon_k &\vdashShareTycon &\E'  \\
  \end{array} \\
  \B,\E',\overline{longstrid}, pairs        \vdashShareStr \E''
  \end{array}}
 {\B,\E, \overline{longstrid}, (i,j)::pairs \vdashShareStr \E''}
\]

\section{Structure}
{\renewcommand{\arraystretch}{1.2}\[
\begin{array}{lcl}
strdec
    & = & dec	                        	\\
    & | & \Local\ strdec\ \In\ strdec\ \End	\\
    & | & strdec\ \key{;}\ strdec			\\
    & | & \Structure\ \overline{strbind}	\\

strbind
    & = & strid = str			\\
    & | & strid :  sigexp = str	\\
    & | & strid :> sigexp = str	\\
    
str 
    & = & longstrid 				\\
    & | & \Struct\ dec\ \End 		\\
    & | & str :  sigexp				\\
    & | & str :> sigexp				\\
    & | & fctid\,(fctarg)			\\
    & | & \Let\ dec\ \In\ str\ \End \\

fctarg
	& = & dec						\\
	& | & str						\\
	    
\end{array}\]}

\section{Elaboration of structure}
\vspace{-25pt}
\begin{flushright}
\framebox{ $ \B, strdec \vdashStrDec coredec, \E, \rho$ }
\end{flushright}
\[
\cfrac
 {\B, dec \vdashD      cdecs, \E, \rho}
 {\B, dec \vdashStrDec cdecs, \E, \rho}
\]

\[
\cfrac
 {         \B             , strdec_1 \vdashStr cdecs_1, \E_1, \rho_1  \qquad
  ([\rho_1]\B) \oplus \E_1, strdec_2 \vdashStr cdecs_2, \E_2, \rho_2  }
 {\B, \Local\ strdec_1\ \In\ strdec_2\ \End \vdashStrDec cdecs_1+cdecs_2, \E_2, \compose{\rho_2}{\rho_1}}
\]

\[
\cfrac
 {         \B             , strdec_1 \vdashStr cdecs_1, \E_1, \rho_1  \qquad
  ([\rho_1]\B) \oplus \E_1, strdec_2 \vdashStr cdecs_2, \E_2, \rho_2  }
 {\B, strdec_1\ \key{;}\ strdec_2 \vdashStrDec cdecs_1 + cdecs_2, ([\rho]\E_1)+\E_2, \compose{\rho_2}{\rho_1}}
\]

\[
\cfrac
 {\B, strbind_i \vdashStrBind cdecs_i, \E_1, \rho_i}
 {\B, \Structure\ strbind \vdashStrDec \sum_i cdecs_i, \sum_i \E_i, \prod_i \rho_i}
\]

\subsection{elaborating structure expression}

\[
\cfrac
 {longstrid \mapsto \E \in^\sSE \B}
 {\B, longstrid \vdashStr \Empty,\E,\Empty}
\]

\[
\cfrac
 {\B dec \vdashD cdecs,\E,\rho}
 {\B, \Struct\ dec\ \End \vdashStr cdecs,\E,\rho}
\]

\[
\cfrac
 {\begin{array}{c}
  \B,sigexp \vdashSigexp \E_1  		 \qquad
  \B,str \vdashStr cdecs,\E_2,\rho	 \\
  \matchEnv[\varphi]{\E_1}{\E_2}	 \qquad	
  \T = \TYNAMES{\E_1} - \TYNAMES{\E\ \of\ \B} \qquad
  \SUPP{\varphi}\subseteq\T
  \end{array}}
 {\B, str:sigexp \vdashStr \GENTYPES{[\varphi]\T}{[\varphi]\E_1}}
\]

\[
\cfrac
 {\begin{array}{c}
  \B,sigexp \vdashSigexp \E_1	     \qquad
  \B,str \vdashStr cdecs,\E_2,\rho	 \\
  \matchEnv[\varphi]{\E_1}{\E_2}	 \qquad	
  \T = \TYNAMES{\E_1} - \TYNAMES{\E\ \of\ \B} \qquad  
  \SUPP{\varphi}\subseteq\T
  \end{array}}
 {\B, str:>sigexp \vdashStr \GENTYPES{\T}{\E_1}}
\]

\[
\cfrac
 {\begin{array}{c}
  \B,fctarg \vdashFctArg cdecs_1, \E, \rho                              	\qquad
  fctid \mapsto (\fctor{\E_1^{\sT_1}}{\E_2^{\sT_2}}, cdecs_2) \in^\sG \B  	\\
  \matchEnv[\varphi]{\E_1}{\E}                                          	\qquad
  \SUPP{\varphi}\subseteq \T_1                                          	\\
  \end{array}}
 {\B,fctid\,(fctarg) \vdashStr cdecs_1 + cdecs_2, [\varphi](\GENTYPES{\T_2}{\E_2}), \rho}
\]

\[
\cfrac
 {\B, dec \vdashD cdecs_1,\E, \rho_1                    \qquad
  [\rho]\B \oplus \E, str \vdashStr cdecs_2, \E', \rho_2}
 {\B, \Let\ dec\ \In\ str\ \End \vdashStr cdecs_1+cdecs_2,\E', \compose{\rho_2}{\rho_1}}
\]

\subsection{elaborating structure bindings}

\[
\cfrac
 {\B, str \vdashStr cdecs, \E, \rho}
 {\B, strid = str \vdashStrBind cdecs, \braced{strid \mapsto \E}^\sSE, \rho}
\]

\[
\cfrac
 {\begin{array}{c}
  \B, str    \vdashStr    cdecs_1, \E_1, \rho    \qquad
  \B, sigexp \vdashSigexp \E_2                   \\
  \T = \TYNAMES{\E_2}-\TYNAMES{\B}               \qquad
  \E'_2 = \GENTYPES{\T}{\E_2}                    \\
  \matchEnv[\varphi]{\E'_2}{\E_1}                \qquad
  \SUPP{\varphi} \subseteq ([\varphi]\T)
  \end{array}}
 {\B, strid: sigexp = str \vdashStrBind cdecs, \braced{strid \mapsto [\varphi]\E'_2}, \rho}
\]

\[
\cfrac
 {\begin{array}{c}
  \B, str    \vdashStr    cdecs_1, \E_1, \rho    \qquad
  \B, sigexp \vdashSigexp \E_2                   \\
  \T = \TYNAMES{\E_2}-\TYNAMES{\B}               \qquad
  \E'_2 = \GENTYPES{\T}{\E_2}                    \\
  \matchEnv[\varphi]{\E'_2}{\E_1}                \qquad
  \SUPP{\varphi} \subseteq ([\varphi]\T)
  \end{array}}
 {\B, strid:>sigexp = str \vdashStrBind cdecs, \braced{strid \mapsto \E'_2}, \rho}
\]

\subsection{elaborating functor argument}
\[
\cfrac
 {\B, str \vdashStr cdecs,\E,\rho}
 {\B, str \vdashFctArg cdecs, \E, \rho}
\]

\[
\cfrac
 {\B, dec \vdashD   cdecs,\E,\rho}
 {\B, dec \vdashFctArg cdecs, \E, \rho}
\]

\section{functor}
{\renewcommand{\arraystretch}{1.2}\[
\begin{array}{lcl}
fctbind
    & = & fctid\,(fctparm) \angled{fctcons} = str \\
fctcons    
    & = & :  sigexp = str \\
    & | & :> sigexp = str \\
fctparm 
    & = & strid: sigexp                  \\
    & | & spec                           \\
\end{array}
\]}

\section{Elaboration of functor}
\vspace{-25pt}
\begin{flushright}
\framebox{ $ \B, fctbind \vdashFctBind \B', \rho$ }
\end{flushright}

\[
\cfrac
 {\begin{array}{c}
  \B,fctparm \vdashFctParm \E_1^{\sT_1}                               \qquad
  \B \oplus \E_1, str \angled{fctcons} \vdashStr cdecs, \E_2, \rho    \\
  \T_2 = \TYNAMES{\E_2}-\T_1-\TYNAMES{\B}                 
  \end{array}}
 {\B, fctid\,(fctparm) \angled{fctcons} = str \vdashFctBind 
	\braced{fctid \mapsto (\fctor{\E_1^{\sT_1}}{\E_2^{\sT_2}}, cdecs)}^\sF, \rho}
\]

\subsection{elaborating functor parameter}

\[
\cfrac
 {\B, sigexp \vdashSigexp \E_1      \qquad
  \T = \TYNAMES{\E_1}-\TYNAMES{\B}  \qquad
  \E_2 = \GENTYPES{\T}{\E_1}}
 {\B, strid: sigexp \vdashFctParm \E_2^{\sT}}
\]

\[
\cfrac
 {\B, spec \vdashSpec \E_1          \qquad
  \T = \TYNAMES{\E_1}-\TYNAMES{\B}  \qquad
  \E_2 = \GENTYPES{\T}{\E_1}}
 {\B, spec \vdashFctParm \E_2^{\sT}}
\]

\section{Program}
{\renewcommand{\arraystretch}{1.2}\[
\begin{array}{lcl}
prog
    & = & strdec                            \\
    & | & \signature\ \overline{sigbind}    \\
    & | & \functor\   \overline{fctbind}	\\
    & | & prog\ \key{;}\ prog				\\
\end{array}
\]}

\section{Elaboration of program}
\vspace{-25pt}
\begin{flushright}
\framebox{ $ \B, prog \vdashProg \B', \rho$ }
\end{flushright}

\[
\frac
 {\B, strdec \vdashStrDec \E,\rho}
 {\B, strdec \vdashProg \E,\rho}
\]

\[
\frac
 {\begin{array}{c}
  \B, sigbind_1 \vdashSigBind \B_1 \\
  \B \oplus \B_1, sigbind_2 \vdashSigBind \B_2 \\
  \cdots \\
  \B \oplus \B_1 \cdots \oplus \B_{n-1}, sigbind_n \vdashSigBind \B_n \\
  \end{array}}
 {\B, \signature\ sigbind_{1..n} \vdashProg \sum_i \B_i, \Empty}
\]

\[
\frac
 {\begin{array}{lll}
  \B      , fctbind_1 \vdashFctBind \B_1,\rho_1  & \B'_1 = ([\rho_1]\B)    + \B_1 & \B''_1 =                  \B_1\\
  \B'_1    , fctbind_2 \vdashFctBind \B_2,\rho_2 & \B'_2 = ([\rho_2]\B'_1) + \B_2 & \B''_2 = ([\rho_2]\B''_1)+\B_2\\
  \multicolumn{3}{c}{\cdots} \\
  \B'_{n-1}, fctbind_n \vdashFctBind \B_n,\rho_n & \wildcard & \B''_n= ([\rho_n]\B''_{n-1}) + \B_{n} \\
  \end{array}}
 {\B, functor\ \overline{fctbind} \vdashProg \B''_n, \prod_i\rho_i}
\]


\[
\cfrac
 {\B               , prog_1 \vdashProg \B_1,\rho_1 \qquad
  [\rho_1]\B + \B_1, prog_2 \vdashProg \B_2,\rho_2 }
 {\B, prog_1\ \key{;}\ prog_2 \vdashProg [\rho_2]\B_1 + \B_2, \compose{\rho_2}{\rho_1}}
\]

\section{realisation}
\vspace{-25pt}
\begin{flushright}
\framebox{ $ \matchEnv[\varphi]{\E_1}{\E_2} $ }
\end{flushright}

\newcommand{\instEop} {:<}
\newcommand{\instE}  [3][]{\ifthenelse{\isempty{#1}}{#2 \instEop #3}{#2 \instEop #3 = #1}}
\newcommand{\enrichE}[2]{#1 \prec #2}
\[
\cfrac
 {\instE[\varphi]{\E_1}{\E_2} \qquad \enrichE{[\varphi]\E_1}{\E_2} }
 {\matchEnv[\varphi]{\E_1}{\E_2}}
\]
\subsection{instantiation}
\[
\cfrac
 {\instE[\varphi_1]{\TE_1}{\TE_2} \qquad \instE[\varphi_2]{\TE_2}{\SE_2}}
 {\instE[\varphi_1 + \varphi_2]{(\TE_1,\SE_1,\VE_1)}{(\TE_2,\SE_2,\VE_2)}}
\]
\[
\cfrac
 {\begin{array}{c}
  tycon_i \mapsto (k_{i_1}, tyfun_{i_1}) \in \TE_1 \qquad tycon_i \mapsto (k_{i_2}, tyfun_{i_2}) \in \TE_2 \\
  k_{i_1} = k_{i_2} \qquad eq(tyfun_{i_1}) = eq(tyfun_{i_2}) \qquad \instE[\varphi_i]{tyfun_{i_1}}{tyfun_{i_2}} \\
  \end{array}}
 {\instE[\sum_i\varphi_i]{\TE_1}{\TE_2}}
\]

\[
\cfrac
 {\begin{array}{c}
  strid_i \mapsto \E_{i_1} \in \SE_1 \qquad strid_i \mapsto \E_{i_2} \in \SE_2 \\
  \instE[\varphi_i]{\E_{i_1}}{\E_{i_2}}
  \end{array}}
 {\instE[\sum_i\varphi_i]{\SE_1}{\SE_2}}
\]

\[
\begin{array}{lclcl}
{\tycon}    & \instEop & {tyfun}     & = & \braced{\tycon \mapsto tyfun} \\
{\wildcard} & \instEop & {\wildcard} & = & \Empty \\
\end{array}
\]

\subsection{enrichment}
\[
\cfrac
 {\enrichE{\TE_1}{\TE_2} \qquad \enrichE{\SE_1}{\SE_2} \qquad \enrichE{\VE_1}{\VE_2}}
 {\enrichE{(\TE_1,\SE_1,\VE_1)}{(\TE_2,\SE_2,\VE_2)}}
\]

\[
\cfrac
 {\begin{array}{cl}
  \multicolumn{2}{c}{\TYNAMES{\TE_1}\subseteq\TYNAMES{\TE_2}}   \\
  \forall\ tycon \in \TYNAMES{\TE_1}. 
    & tycon \mapsto tystr_1 \in \TE_1    \\
    & tycon \mapsto tystr_2 \in \TE_2    \\
    & \enrichE{tystr_1}{tystr_2}
  \end{array}}
 {\enrichE{\TE_1}{\TE_2}}
\]

\[
\cfrac
 {cons_1 = \emptyset \vee cons_1 = cons_2}
 {\enrichE{(k, tyfun, cons_1)}{(k, tyfun, cons_2)}}
\]

\[
\cfrac
 {\begin{array}{cl}
  \multicolumn{2}{c}{\KEYS{\SE_1} \subseteq \KEYS{\SE_2}}   \\
  \forall\ strid \in \KEYS{\SE_1}. 
    & strid \mapsto \E_1 \in \SE_1 \\
    & strid \mapsto \E_2 \in \SE_2 \\
    & \enrichE{\E_1}{\E_2}
  \end{array}}
 {\enrichE{\SE_1}{\SE_2}}
\]

\[
\cfrac
 {\begin{array}{cl}
  \multicolumn{2}{c}{\KEYS{\VE_1} \subseteq \KEYS{\VE_2}}   \\
  \forall\ vid \in \KEYS{\VE_1}. 
    & vid \mapsto val_1 \in \VE_1 \\
    & vid \mapsto val_2 \in \VE_2 \\
    & \enrichE{val_1}{val_2}
  \end{array}}
 {\enrichE{\VE_1}{\VE_2}}
\]

\[
\cfrac
 {kind_1 = \VKV \vee kind_1 = kind_2}
 {\enrichE{(\absval,\delta,kind_1)}{(\wildcard, \delta, kind_2)}}
\]

\end{document}

